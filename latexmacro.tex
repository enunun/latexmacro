\documentclass[10pt, a5j, ]{ltjsbook} % lualatex

\usepackage{caption}
\usepackage{subcaption}
\renewcommand{\thesubfigure}{\alph{subfigure}}

\makeatletter
\captionsetup{compatibility=false}
\captionsetup{%
	format = plain,
	labelsep = quad,
	font = small,
	skip = .2\baselineskip,
	width = .85\linewidth,
	subrefformat = parens,
	skip = 5\jsc@mpt
}
\captionsetup[subfigure]{%
	labelformat = parens,
	labelsep = space
}
\makeatother

%\NewDocumentCommand\paren{sm}{%
    \IfBooleanTF{#1}{%
        \mathopen{}(#2)\mathclose{}%
    }%
    {%
        \mathopen{}\left( #2 \right)\mathclose{}%    
    }%
}
\NewDocumentCommand\apply{smm}{%
    \IfBooleanTF{#1}{%
        #2\paren*{#3}%
    }%
    {%
        #2\paren{#3}
    }%
}

%%%%%%%%=========package=============%%%%%%%%%%%%%
%
\usepackage[no-math]{luatexja-fontspec}
\usepackage{fontspec}
\usepackage{unicode-math}
\setmainfont[Ligatures=TeX]{Libertinus Serif}
\setsansfont[Ligatures=TeX]{Libertinus Sans}
\setmathfont{Libertinus Math}[math-style=ISO, bold-style=ISO]
\setmathfont{TeX Gyre Pagella Math}[range=bb]
\DeclareFontFamily{OML}{nxlmi}{\skewchar \font =127}
\DeclareFontShape{OML}{nxlmi}{m}{it}{
	<-6.3>  nxlmi035
	<6.3-8.6> nxlmi037
	<8.6->  nxlmi03
}{}
\DeclareSymbolFont{gletters}{OML}{nxlmi}{m}{it}
\DeclareMathSymbol{g}{\mathalpha}{gletters}{`g}
%\setmonofont{HackGen}
%\setmonojfont{HackGen}
\usepackage[HaranoAji]{luatexja-preset}
\ltjsetparameter{jacharrange={-2,-3}}
\usepackage{amsthm} 
\usepackage[draft]{graphicx} 
\usepackage{booktabs} 
\usepackage{diffcoeff} 
\usepackage{wrapfig} 
\usepackage{enumitem} 
\usepackage{multienum} 
\usepackage[binary-units = true]{siunitx} 
\usepackage{index} 
\usepackage{comment} 
\usepackage{framed} 
\usepackage{listings} 
\usepackage{pxrubrica}
\usepackage{luatexja-ruby}
\usepackage{tikz}
\usepackage{xcolor}
\usepackage[listings, breakable]{tcolorbox}
\usepackage{dirtree}
\usepackage[edges]{forest}
\forestset{%
	for forest={graw'=0, draw, folder}%
}

\usepackage{bxtexlogo} 
\bxtexlogoimport{*,**}

\delimitershortfall = -.5pt 
%
\graphicspath{{./fig/}}
%
%
%%%%%%%%%%%===================enumerate==================%%%%%%%%%%%
%
\setlist[description, 1]{%
  style = nextline%
}
%
%%%%%%%%%%%%%%=======================theorem==================%%%%%%%%%%%%%
%
%
\theoremstyle{definition}
\newtheorem{Ex}{例}[section]
\newtheorem{Note}{注意}[section]
%
%%%%%%%%%%%====================== logo =================%%%%%%%%%%%%
%
\providecommand\upLaTeX{up\LaTeX}
\providecommand\TeXLive{{\TeX} Live}
\providecommand\WTeX{W32\TeX}
\providecommand\MacTeX{Mac\TeX}
%
%%%%%%%%%%%%%%%=============index=====================%%%%%%%%%%%%%
%
\newindex{cidx}{cidx}{cind}{コマンド索引}
\newindex{pidx}{pidx}{pind}{パッケージ・クラス索引}
\newindex{eidx}{eidx}{eind}{環境索引}
\newindex{widx}{widx}{wind}{用語索引}
%
%
%
%%%%%%%%%%%%%================== name ===============%%%%%%%%%%%%%
%
%
\NewDocumentCommand\Word{}{Word}
\NewDocumentCommand\MSWord{}{MS Word}
\NewDocumentCommand\LOWriter{}{LibreOffice Writer}
\NewDocumentCommand\ichitarou{}{一太郎}
\NewDocumentCommand\indesign{}{InDesign}
\NewDocumentCommand\neovim{}{Neovim}
\NewDocumentCommand\roff{}{roff}
\NewDocumentCommand\html{}{HTML}
\NewDocumentCommand\css{}{CSS}
\ProvideDocumentCommand\SATySFi{}{SATySFi}
\NewDocumentCommand\cpp{}{%
  C\nolinebreak\hspace{-.05em}\raisebox{.18ex}{\textbf{+}}%
  \nolinebreak\hspace{-.14em}\raisebox{.18ex}{\textbf{+}}%
}

\NewDocumentCommand\Windows{}{Windows}
\NewDocumentCommand\MacOS{}{MacOS}
\NewDocumentCommand\Linux{}{Linux}
\NewDocumentCommand\UNIX{}{UNIX}

\NewDocumentCommand\Excel{}{Excel}
\NewDocumentCommand\tlmgr{}{tlmgr}
\NewDocumentCommand\PowerShell{}{PowerShell}
\NewDocumentCommand\WSL{}{WSL}
\NewDocumentCommand\TeXworks{}{{\TeX}Works}
\NewDocumentCommand\TeXShop{}{{\TeX}Shop}
\NewDocumentCommand\TeXDS{}{{\TeX} Directory Structure}
\NewDocumentCommand\TDS{}{TDS}
\NewDocumentCommand\TeXUsersGroup{}{{\TeX} Users Group}
\NewDocumentCommand\TUG{}{TUG}
\NewDocumentCommand\CTeXArchive{}{Comprehensive {\TeX} Archive Network}
\NewDocumentCommand\CTAN{}{CTAN}
\NewDocumentCommand\LaTeXProject{}{{\LaTeX} Project}
\NewDocumentCommand\TeXdoc{}{{\TeX}doc}
\NewDocumentCommand\Kpathsea{}{Kpathsea}
\NewDocumentCommand\lsR{}{\lstinline{ls-R}}
\NewDocumentCommand\mktexlsr{}{mktexlsr}
\NewDocumentCommand\GhostScript{}{GhostScript}

\NewDocumentCommand\VStudioCode{}{Visual Studio Code}
\NewDocumentCommand\VSCode{}{VS Code}
\NewDocumentCommand\Vim{}{Vim}
\NewDocumentCommand\SublimeText{}{Sublime Text}
\NewDocumentCommand\Atom{}{Atom}
\NewDocumentCommand\Acrobat{}{Adobe Acrobat Reader DC}
\NewDocumentCommand\Evince{}{Evince}
\NewDocumentCommand\Okular{}{Okular}
\NewDocumentCommand\Skim{}{Skim}
\NewDocumentCommand\SumatraPDF{}{SumatraPDF}
\NewDocumentCommand\Inkscape{}{Inkscape}
\NewDocumentCommand\gnuplot{}{gnuplot}
\NewDocumentCommand\llmk{}{llmk}
%
%%%%%%%%%%%%%%%%%%=====================macro==================%%%%%%%%%%%%%%%%%%%%%%
%
\NewDocumentCommand\bs{}{\textbackslash}
\NewDocumentCommand\mathbstt{}{$\fontspec{HackGen}\text{\textbackslash}$}
\NewDocumentCommand\command{m}{%
	\mathbstt%
	$\text{\texttt{#1}}$%
}
\NewDocumentCommand\mathasttt{}{$\fontspec{HackGen}\text{{*}}$}
\NewDocumentCommand\option{m}{\texttt{#1}}

\NewDocumentCommand\cs{s m}{%
	\IfBooleanTF{#1}{%
		\command{#2}\mathasttt%
	}{%
	\command{#2}%
	}%
}
\NewDocumentCommand\environment{m}{\texttt{#1}}
\NewDocumentCommand\env{m}{\environment{#1}}
\NewDocumentCommand\package{m}{\texttt{#1}}
\NewDocumentCommand\ltxclass{m}{\texttt{#1}}
\NewDocumentCommand\term{m}{\textgt{\textbf{#1}}}
%\RenewDocumentCommand{\emph}{m}{\textbf{\textgt{#1}}}
\NewDocumentCommand\envvar{m}{\texttt{#1}}
\NewDocumentCommand\dir{m}{$\text{\lstinline{#1}}$}
\NewDocumentCommand\cmd{m}{\texttt{#1}}
\NewDocumentCommand\register{m}{\cs{#1}}
\NewDocumentCommand\filename{m}{\texttt{#1}}
\NewDocumentCommand\meta{m}{$\langle${\normalfont\itshape#1}$\rangle$}
\NewDocumentCommand\marg{m}{$\fontspec{HackGen}\text{\textbraceleft\texttt{#1}\textbraceright}$}
\NewDocumentCommand\metamarg{m}{$\fontspec{HackGen}\text{\textbraceleft\meta{#1}\textbraceright}$}
\NewDocumentCommand\oarg{m}{$\fontspec{HackGen}\text{[\texttt{#1}]}$}
\NewDocumentCommand\metaoarg{m}{$\fontspec{HackGen}\text{[\meta{#1}]}$}
\NewDocumentCommand\parg{m}{$\fontspec{HackGen}\text{\texttt{(#1)}}$}
\NewDocumentCommand\metaparg{m}{$\fontspec{HackGen}\text{\texttt{(\meta{#1})}}$}
\NewDocumentCommand\Unicodenumber{m}{\texttt{U+#1}}
\NewDocumentCommand\keyvaluemethod{}{\meta{key}{=}\meta{value}}
\NewDocumentCommand\key{m}{\option{#1}}

\NewDocumentEnvironment{directorytree}{}{\center\forest for tree={grow'=0, draw, folder, font=\ttfamily}}{\endforest\endcenter}


%%%%%%%%%%%===========listings===============%%%%%%%%%%%%%%%%

\lstdefinestyle{latexsample}{%
  language={[LaTeX]TeX},%
  frame=none,
  basicstyle={\footnotesize\ttfamily},%
  identifierstyle={\small\ttfamily},%
  commentstyle={\small\ttfamily},%
  keywordstyle={\small\ttfamily},%
  stringstyle={\small\ttfamily},%
  breaklines=true,
  numbers=left,%
  numbersep=6pt,
  numberstyle={\footnotesize\upshape\ttfamily},%
  escapeinside="",%
  breakindent=0pt,
  tabsize = 2
}%

\lstset{%
	basicstyle = {\ttfamily},%
	tabsize = 4,%
}

\newlength\defaultindent
\setlength\defaultindent{\parindent}

\newtcbinputlisting[]{\sampleinput}[2][]{%
	listing file =#2, listing options = {style={latexsample},},
	colback = white,
	before lower = {\setlength\parindent{\defaultindent}}, after upper = {\vspace{-1.5mm}\tcbsubtitle{Output}},%
	breakable, toprule at break = 0mm, bottomrule at break = 0mm, topsep at break = 2mm,%
	title = {Input}, lower separated = false, middle = 0mm, top=-1mm, boxsep=1mm, leftlower = 1mm, right = 1mm, bottom = 1mm,%
	sharp corners = all, after={}, #1%
}

\newtcbinputlisting[]{\inputonly}[2][]{%
	listing file =#2, listing options = {style=latexsample}, listing only, colback = white,%
	before lower = {\setlength\parindent{\defaultindent}}, after upper = {},%
	breakable, toprule at break = 0mm, bottomrule at break = 0mm, topsep at break = 2mm,%
	title = {Source Code}, lower separated = false, middle = 0mm, top=-1mm, boxsep=1mm, leftlower = 1mm, right = 1mm, bottom = -1mm,%
	sharp corners = all, after={}, #1%
}

\makeatletter
\newtcbinputlisting[]{\sampleinputspacevisible}[2][]{%
	listing file =#2, listing options = {style={latexsample},showspaces=true, showtabs = true,, tabsize = 4},
	colback = white,
	before upper = {\def\lst@visiblespace{\normalfont\textvisiblespace}\ignorespaces},%
	before lower = {\setlength\parindent{\defaultindent}}, after upper = {\vspace{-1.5mm}\tcbsubtitle{Output}},%
	breakable, toprule at break = 0mm, bottomrule at break = 0mm, topsep at break = 2mm,%
	title = {Input}, lower separated = false, middle = 0mm, top=-1mm, boxsep=1mm, leftlower = 1mm, right = 1mm, bottom = 1mm,%
	sharp corners = all, after={}, #1%
}

\newtcbinputlisting[]{\inputonlyspacevisible}[2][]{%
	listing file =#2, listing options = {style={latexsample}, showspaces = true, showtabs = true}, listing only, colback = white,%
	before = {\def\lst@visiblespace{\textvisiblespace}},%
	before lower = {\setlength\parindent{\defaultindent}}, after upper = {},%
	breakable, toprule at break = 0mm, bottomrule at break = 0mm, topsep at break = 2mm,%
	title = {Source Code}, lower separated = false, middle = 0mm, top=-1mm, boxsep=1mm, leftlower = 1mm, right = 1mm, bottom = -1mm,%
	sharp corners = all, after={}, #1%
}
\makeatother

\lstdefinestyle{console}{%
  language=sh,%
  basicstyle={\color{white}\footnotesize\ttfamily},%
  identifierstyle={\small},%
  commentstyle={\small},%
  keywordstyle={\small},%
  stringstyle={\small},%
  frame={single},%
  breaklines=true,
  backgroundcolor = \color{black},
  escapeinside={!!},
  showtabs = false,
  tabsize=1,
}


\newtcblisting{console}{%
	listing options={style=console},%
	listing only, every listing line = {\small\footnotesize\$},%
	boxrule=0mm,
	colback = black, colframe = black, coltext = white, sharp corners = all, after = {}, frame empty,%
	middle = 0mm, top=0mm, boxsep=0mm, left = 0mm, right = 0mm, bottom = -3mm,%
}

\makeatletter
\@addtoreset{footnote}{section}
\makeatother

%%
%%%%%%%%%%%%%%========hyperref============%%%%%%%%%%
%
\usepackage[unicode, pdfusetitle, hidelinks, draft = true]{hyperref} % hyperlink

\hypersetup{% setting hyperref
  bookmarksnumbered=true,
  bookmarksopen=true,
  bookmarkstype=toc,
  pdfborder={0 0 0},
  colorlinks = true,
}
\usepackage{footnotebackref}
%
%
%%%%%%%==============crossreference================%%%%%%%%%%
%
\usepackage[nameinlink]{cleveref}
\crefname{figure}{図}{図}
\Crefname{figure}{図}{図}
\crefformat{chapter}{#2第#1章#3}
\Crefformat{chapter}{#2第#1章#3}
\crefrangeformat{chapter}{#3第#1#4章から#5第#2#6章}
\Crefrangeformat{chapter}{#3第#1#4章から#5第#2#6章}
\crefname{page}{p.}{p.}
\Crefname{page}{p.}{p.}
\crefname{Ex}{例}{例}
\Crefname{Ex}{例}{例}
\crefname{table}{表}{表}
\Crefname{table}{表}{表}
\crefname{section}{\presectionname}{\presectionname}
\Crefname{section}{\presectionname}{\presectionname}
\crefname{subsection}{\presubsectionname}{\presubsectionname}
\Crefname{subsection}{\presubsectionname}{\presubsectionname}
\crefrangeformat{section}{#3\presectionname#1#4から#5\presectionname#2#6}
\Crefrangeformat{section}{#3\presectionname#1#4から#5\presectionname#2#6}
\crefrangeformat{subsection}{#3\presubsectionname#1#4から#5\presubsectionname#2#6}
\Crefrangeformat{subsection}{#3\presubsectionname#1#4から#5\presubsectionname#2#6}

%
%
%%%%%%%======biblatex=============%%%%%%%%%%
%
\usepackage[%
backend = biber,%
url = false,%
doi = false,%
eprint = false,%
isbn = false,%
style = numeric-comp%
]{biblatex} % use "biblatex"
\DeclareFieldFormat*{title}{\mkbibquote{#1}}
\DeclareDelimFormat{labelnamepunct}{\addcomma\addspace}
\DeclareFieldFormat{url}{\url{#1}}
\addbibresource{reference/book.bib}
%\addbibresource{reference/online.bib}

\makeatletter
\renewcommand{\section}{%
  \if@slide\clearpage\fi
  \@startsection{section}{1}{\z@}%
  {\Cvs \@plus.5\Cdp \@minus.2\Cdp}% 前アキ
  {.5\Cvs \@plus.3\Cdp}% 後アキ
  %   {\normalfont\Large\headfont\@secapp}}
  {\normalfont\Large\headfont\presectionname\raggedright}%
}
\renewcommand{\subsection}{%
  \@startsection{subsection}{2}{\z@}%
  {\Cvs \@plus.5\Cdp \@minus.2\Cdp}% 前アキ
  {.5\Cvs \@plus.3\Cdp}% 後アキ
  {\normalfont\large\headfont\presubsectionname}%
}
\makeatother
\renewcommand{\presectionname}{\S}
\newcommand{\presubsectionname}{\S\S}
\setcounter{tocdepth}{2}
%
%


%%%%%%%%%%%%%%%===========index============%%%%%%%%%%%

\NewDocumentCommand\symbolindexname{}{記号}

\usepackage{needspace}

\makeatletter
\NewDocumentCommand\idxhead{m}{%
  \needspace{2\baselineskip}%
  \vspace{\baselineskip}%
  \def\hrulefill{\leavevmode\leaders\hrule height 0.8pt\hfill\kern\z@}%
  \hbox to \columnwidth{\hfil%
    \normalsize%
    \textcolor{gray}{\raisebox{.15ex}{■}}\hspace{.5\zw}\textsf{#1}\hspace{.5\zw}\textcolor{gray}{\raisebox{.15ex}{■}}\hfil}\vspace{-4.5mm}\hrulefill\par%
  \nopagebreak%
}
\makeatother
%


%\includeonly{contents/markup}
%\includeonly{contents/macro}



\begin{document}
%
%\begin{titlepage}
%	\title{構造化マークアップ志向の{\LaTeX}入門}
%	\author{野口匠}
%	\date{2022年12月27日}
%	\maketitle
%\end{titlepage}
%
%
%
\frontmatter % introduction
%
\chapter{はじめに}

現在,{\TeX}/{\LaTeX}は自然科学,特に数学や物理学などの数式が多く含まれるような
分野において,レポートや論文のような文書を作成する際には事実上必須のツールとなっている.
しかも,レポートや論文だけでなくセミナーの資料やちょっとしたネット上の解説記事も
{\TeX}/{\LaTeX}で作成されることが多くなっている.
このように,{\TeX}/{\LaTeX}は特に自然科学を学ぶ者にとっては馴染みのあるツールであり,
当該分野の知識に加えてその使い方を習得する必要に駆られている.

では彼らがどのように{\TeX}/{\LaTeX}の使い方を習得するかといえば,
広く出版されている入門書を手に取る場合もあるけれども,
おおよそ研究室やサークルの先輩からの口伝や,
インターネット上の解説記事を頼りにした独学が多いのではないだろうか.
もちろん彼らは{\TeX}/{\LaTeX}のエキスパートになりたいわけではなく,
自分の要求に堪えるレベルで文書が作成できればそれで十分なのではあるが,
そのためかあまり推奨されないような使い方をしている場面にしばしば遭遇する.
具体例を挙げると,
\begin{itemize}
	\item \cs{quad}や\cs{textbf}などの「見た目系コマンド」の本文中での多用,
	\item \ltxclass{jarticle}や\ltxclass{jreport}などの時代遅れのドキュメントクラスの利用,
	\item \env{eqnarray}環境や\lstinline{$$...$$}のような時代遅れの数式環境の利用
\end{itemize}
などである.

これらの問題のうち,後者2つに関しては解説を見かける機会は非常に多い.
時代遅れのドキュメントクラスや環境の使用は,出力がおかしくなったりそもそもタイプセットができないといった
致命的な問題を引き起こすからであろう.
しかし,1つ目の問題に対しては後者2つに比べれば解説を見かけることは多くない.
「見た目系コマンド」が本文中で多用されたからといって,出力がおかしくなるような致命的な問題は生じないからであろう.

{\TeX}/{\LaTeX}はマークアップシステムであるから,文書の意味内容を表す「構造」とそれらの構造がどのように紙面に表現されるかの「体裁」は
できる限り分離されるべきである.
この「構造と体裁をできる限り分離してソースを書こう」という姿勢でなされるマークアップは構造化マークアップと呼ばれ,
特にWebページを作成する際には必須とみなされている%
\footnote{%
Webページを作成する際に構造化マークアップが必須となるのはブラウザの多様化や検索エンジンとの関係が大きく,
筆者が{\LaTeX}で構造化マークアップを勧める理由とはかなり異なるが.
}%
.

本書は{\LaTeX}界隈にもこの流れを取り込むべく執筆された本である.
具体的には,{\TeX}/{\LaTeX}の基本的な事項を習得した初心者が
構造化マークアップを意識した美しいコードを書けるようになることが目標である.
従って,{\TeX}/{\LaTeX}の基本的な事項は習得済みであることを想定している.
{\TeX}/{\LaTeX}でレポートや論文を作成した経験があればあまり問題ないと思われる.
まずは新しいコマンドや環境の作成方法を述べ,
それを使って文章や数式の意味内容を効果的に表現する例を挙げる.

本書で挙げたテクニックを習得できれば,ソースコード中のコマンドの意図が分かりやすくなり可読性が向上するだけでなく,
汎用的に使えるコマンドを切り出して別ファイルに保存し,それを別の文書に使いまわすことも容易となる%
.本書の読者がこのような恩恵を実感できるようになれば何よりである.

本書では,環境に対する依存性が低い話題が中心であるが,具体的な実装面では2022年12月ごろにインストールされた{\TeXLive} 2022を想定する.
それでも,違いが生じうるのはおおよそ「\package{xparse}パッケージを読み込む必要があるかどうか」程度であろう.

なお,本書ではソース中の空白文字を可視化するため,必要な場合には「\cs{\textvisiblespace}」という記法で空白文字を表現することがある.
また,ソース中で「\cs{textbf}\metamarg{argument}」のように山括弧で囲まれたイタリック体の文字が登場した場合,
それは本当に「\lstinline|<argument>|」と記述するわけではなく,状況に応じて適切なものに置き換えることを意図して書かれている.
このことに注意しながら読み進めてほしい.
%
\tableofcontents
%
%
\mainmatter
%
\chapter{予備知識} \label{chap:markup}

\section{構造化マークアップ} \label{sec:markup}

マークアップ方式文書を作成する際,文書の論理的構造とそれを紙面に表示する際の視覚的効果(すなわち体裁)は,
明確に分離して記述するのが望ましいとされることが多い.
それを強調する意味で,マークアップを特に%
\term{構造化マークアップ}%
\index[widx]{こうぞうかマークアップ@構造化マークアップ}%
と呼ぶことがある.
構造化マークアップが重要視される理由は文脈によってさまざまであるが,{\LaTeX}と関係するのはおおむね以下のようなものである:
\begin{itemize}
    \item 文書の体裁を定義する部分が本文とは分離されているため,体裁のみの変更が容易となる.
    \item あらかじめ体裁を定義する部分を切り出してパッケージ化しておくことにより,ユーザーが自ら体裁を定義する必要がなくなる.
\end{itemize}
特に後者は重要である.
体裁の定義は処理系の低レベルな部分に触れなければできないことも多く,その処理系に対する深い知識が要求されることが多い.
そのような知識を一般ユーザーに求めるようでは,そのツールは不便であると言わざるを得ない.

{\LaTeX}では,基本的にドキュメントクラス内で文書の全体的な体裁が定義されている.
有志がドキュメントクラスを開発してくれるおかげで,
ユーザーは文書の論理的構造をソースに記述するだけで「いい感じ」の文書が作成できるのである.
このことが理由で{\LaTeX}を使用するユーザーも決して少なくはないだろう.
他にも,
\term{パッケージ}%
\index[widx]{パッケージ@パッケージ}%
\footnote{%
ここでは,
\cs{usepackage}%
コマンドで読み込むことができるパッケージファイルのことを指している.
}%
という形で提供される拡張機能を読み込むことにより,
{\LaTeX}をさらに便利に使うことができるのは周知のとおりである.

\section{コマンドの定義による構造化マークアップ} \label{sec:exampleinlatex}

さて,与えられたドキュメントクラスやパッケージを活用するだけでは,自分のニーズを完璧に満たせるとは限らない.
そのニーズ自体が{\LaTeX}を使用するうえで不適切である場合%
\footnote{%
例えば,図を自分の思う通りの場所に配置したいというニーズはそもそも組版上不適切であることが多い.
}%
を除けば,新しいコマンドや環境を定義したり,既存のコマンドや環境の定義を変更することで解決されるべきである.
もちろん,技術的な難易度の問題で解決できないこともあるが,簡単に解決できることも多い.
例を挙げよう.

\begin{Ex} \label{Ex:cpp}
    プログラミング言語「{\cpp}」のロゴを出力したいとき,
    そのまま「\lstinline|C++|」と入力しても(Latin Modern Romanフォント採用時には)「{\fontspec{Latin Modern Roman}C++}」と表示されてしまい,
    非常に不格好である.
    
    採用するフォントや好みによるところも大きいが,例えば
    \inputonly{samplecode/cpp}
    のようにいくらかの微調整を加えてやることで,出力が(Latin Modern Romanフォント採用時には)
    「{\fontspec{Latin Modern Roman}C\nolinebreak\hspace{-.04em}%
\raisebox{.4ex}{\tiny\textbf{+}}%
\nolinebreak\hspace{-.14em}%
\raisebox{.4ex}{\tiny\textbf{+}}%}」のようになり,いくらか改善する.
    これを本文に適用したいところであるが,まさかソース中の該当箇所をそのままこれに置き換えようなどとは思うまい.
    置き換えた後に「微調整」の内容を変更した際,ソース中の該当箇所すべてを変更する必要が生じるからである.
    当然修正漏れの可能性もある.例えば「\cs{cpp}」のようなコマンドを上記微調整の結果として定義し,
    本文中では「\cs{cpp}」のように記述すべきである.
\end{Ex}

\cref{Ex:cpp}で挙げたのは,必要とされた特殊な装飾を本文中から外に隔離する例である.
このようにすることで,本文中では「\cs{cpp}」のように,すっきりした記述にできる.

重要なのはそれだけではない.本文中での記述から\cs{hspace}や\cs{raisebox}といった
「余計なもの」が排除され,「\cs{cpp}」という「意図」のみになっている.
これにより,ソースを読むだけで「何を書きたいのか」がすぐにわかる.
書いてすぐならば多少ごちゃごちゃした記述であっても意図を忘れることは多くないだろうが,
時間が経った後に自分が書いたソースを見返した場合に何が書いてあるのかまったくわからなくなる.
「そのようなことを考えるのは面倒だしどうせ見返すようなことなどないのだから,ソースは書き捨てでよい」などと
考えてはいけない.このようなtipsは定期的に再利用したくなるものである.

「意図をはっきりさせる」ことは,異なる意図ではあるが結果的に体裁が同じになった場合に特に有用である.

\begin{Ex} \label{Ex:intention}
    新出の単語も強調も太字で書くことはよくある.強調の場合は標準で用意されている\cs{emph}コマンドを使うべきである.
    ここで,新出の単語も\cs{emph}コマンドを使ってしまうと,ソース中で両者を見分けることができなくなってしまう.
    そのため,新出の単語を示すために例えば\cs{term}といったコマンドを用意することで,両者を区別することが容易になる.

    ただし,「新出の単語を太字表記するのは強調のためで,そこに意味の違いはない.
    その証拠に,新出の単語であっても重要でないものは太字表記しないではないか」と考えるのであれば,
    両者を区別せずに\cs{emph}コマンドで統一するのは理にかなっている.
\end{Ex}

さて,\cref{Ex:intention}で取り上げた\cs{emph}コマンド%
\index[cidx]{emph@\cs{emph}}%
に関して,誤ったtipsが広まってしまっているようである.

\begin{Ex} \label{Ex:changedef}
    強調の際には\cs{emph}というコマンドを使う.ところが,
    この\cs{emph}コマンドは(環境にもよるが)
    「和文はゴシック体にするが,欧文はイタリック体にする」という挙動を示す.
    \inputonly{samplecode/emph}
    とすれば「\emph{なんか重要そうな和文}や\emph{important contents}とか」のような出力が得られる. 
    
    この挙動を嫌がったせいか「強調には直接\cs{textbf}コマンドを使ってしまえ」という誤ったtipsが横行している.
    この場合,当然\cs{emph}コマンドの定義を変更するのが正しい.
\end{Ex}

以降,これを{\LaTeX}で実装するための具体的な方法を考えよう.
%\chapter{予備知識} \label{chap:markup}

\section{構造化マークアップ} \label{sec:markup}

マークアップ方式文書を作成する際,文書の論理的構造とそれを紙面に表示する際の視覚的効果(すなわち体裁)は,
明確に分離して記述するのが望ましいとされることが多い.
それを強調する意味で,マークアップを特に%
\term{構造化マークアップ}%
\index[widx]{こうぞうかマークアップ@構造化マークアップ}%
と呼ぶことがある.
構造化マークアップが重要視される理由は文脈によってさまざまであるが,{\LaTeX}と関係するのはおおむね以下のようなものである:
\begin{itemize}
    \item 文書の体裁を定義する部分が本文とは分離されているため,体裁のみの変更が容易となる.
    \item あらかじめ体裁を定義する部分を切り出してパッケージ化しておくことにより,ユーザーが自ら体裁を定義する必要がなくなる.
\end{itemize}
特に後者は重要である.
体裁の定義は処理系の低レベルな部分に触れなければできないことも多く,その処理系に対する深い知識が要求されることが多い.
そのような知識を一般ユーザーに求めるようでは,そのツールは不便であると言わざるを得ない.

{\LaTeX}では,基本的にドキュメントクラス内で文書の全体的な体裁が定義されている.
有志がドキュメントクラスを開発してくれるおかげで,
ユーザーは文書の論理的構造をソースに記述するだけで「いい感じ」の文書が作成できるのである.
このことが理由で{\LaTeX}を使用するユーザーも決して少なくはないだろう.
他にも,
\term{パッケージ}%
\index[widx]{パッケージ@パッケージ}%
\footnote{%
ここでは,
\cs{usepackage}%
コマンドで読み込むことができるパッケージファイルのことを指している.
}%
という形で提供される拡張機能を読み込むことにより,
{\LaTeX}をさらに便利に使うことができるのは周知のとおりである.

\section{コマンドの定義による構造化マークアップ} \label{sec:exampleinlatex}

さて,与えられたドキュメントクラスやパッケージを活用するだけでは,自分のニーズを完璧に満たせるとは限らない.
そのニーズ自体が{\LaTeX}を使用するうえで不適切である場合%
\footnote{%
例えば,図を自分の思う通りの場所に配置したいというニーズはそもそも組版上不適切であることが多い.
}%
を除けば,新しいコマンドや環境を定義したり,既存のコマンドや環境の定義を変更することで解決されるべきである.
もちろん,技術的な難易度の問題で解決できないこともあるが,簡単に解決できることも多い.
例を挙げよう.

\begin{Ex} \label{Ex:cpp}
    プログラミング言語「{\cpp}」のロゴを出力したいとき,
    そのまま「\lstinline|C++|」と入力しても(Latin Modern Romanフォント採用時には)「{\fontspec{Latin Modern Roman}C++}」と表示されてしまい,
    非常に不格好である.
    
    採用するフォントや好みによるところも大きいが,例えば
    \inputonly{samplecode/cpp}
    のようにいくらかの微調整を加えてやることで,出力が(Latin Modern Romanフォント採用時には)
    「{\fontspec{Latin Modern Roman}C\nolinebreak\hspace{-.04em}%
\raisebox{.4ex}{\tiny\textbf{+}}%
\nolinebreak\hspace{-.14em}%
\raisebox{.4ex}{\tiny\textbf{+}}%}」のようになり,いくらか改善する.
    これを本文に適用したいところであるが,まさかソース中の該当箇所をそのままこれに置き換えようなどとは思うまい.
    置き換えた後に「微調整」の内容を変更した際,ソース中の該当箇所すべてを変更する必要が生じるからである.
    当然修正漏れの可能性もある.例えば「\cs{cpp}」のようなコマンドを上記微調整の結果として定義し,
    本文中では「\cs{cpp}」のように記述すべきである.
\end{Ex}

\cref{Ex:cpp}で挙げたのは,必要とされた特殊な装飾を本文中から外に隔離する例である.
このようにすることで,本文中では「\cs{cpp}」のように,すっきりした記述にできる.

重要なのはそれだけではない.本文中での記述から\cs{hspace}や\cs{raisebox}といった
「余計なもの」が排除され,「\cs{cpp}」という「意図」のみになっている.
これにより,ソースを読むだけで「何を書きたいのか」がすぐにわかる.
書いてすぐならば多少ごちゃごちゃした記述であっても意図を忘れることは多くないだろうが,
時間が経った後に自分が書いたソースを見返した場合に何が書いてあるのかまったくわからなくなる.
「そのようなことを考えるのは面倒だしどうせ見返すようなことなどないのだから,ソースは書き捨てでよい」などと
考えてはいけない.このようなtipsは定期的に再利用したくなるものである.

「意図をはっきりさせる」ことは,異なる意図ではあるが結果的に体裁が同じになった場合に特に有用である.

\begin{Ex} \label{Ex:intention}
    新出の単語も強調も太字で書くことはよくある.強調の場合は標準で用意されている\cs{emph}コマンドを使うべきである.
    ここで,新出の単語も\cs{emph}コマンドを使ってしまうと,ソース中で両者を見分けることができなくなってしまう.
    そのため,新出の単語を示すために例えば\cs{term}といったコマンドを用意することで,両者を区別することが容易になる.

    ただし,「新出の単語を太字表記するのは強調のためで,そこに意味の違いはない.
    その証拠に,新出の単語であっても重要でないものは太字表記しないではないか」と考えるのであれば,
    両者を区別せずに\cs{emph}コマンドで統一するのは理にかなっている.
\end{Ex}

さて,\cref{Ex:intention}で取り上げた\cs{emph}コマンド%
\index[cidx]{emph@\cs{emph}}%
に関して,誤ったtipsが広まってしまっているようである.

\begin{Ex} \label{Ex:changedef}
    強調の際には\cs{emph}というコマンドを使う.ところが,
    この\cs{emph}コマンドは(環境にもよるが)
    「和文はゴシック体にするが,欧文はイタリック体にする」という挙動を示す.
    \inputonly{samplecode/emph}
    とすれば「\emph{なんか重要そうな和文}や\emph{important contents}とか」のような出力が得られる. 
    
    この挙動を嫌がったせいか「強調には直接\cs{textbf}コマンドを使ってしまえ」という誤ったtipsが横行している.
    この場合,当然\cs{emph}コマンドの定義を変更するのが正しい.
\end{Ex}

以降,これを{\LaTeX}で実装するための具体的な方法を考えよう.
\chapter{{\TeX}の字句解析超入門} \label{sec:seq}

マクロの具体的な作り方の前に,まずは{\TeX}の字句解析についてざっくりと復習しておこう.
なお,ここではざっくりとした感覚をつかむことを目的としているため,正確性はまるで意識していない.
そのため,この節の内容は{\TeX}処理系や{\TeX}言語の入門としてはまったく役に立たないということに注意しよう.
むしろ,「この後の説明のどこが破綻しているか」を考えることがよい勉強になる可能性すらある.

\section{プリミティブとマクロ} \label{sec:primitive}

{\TeX}において,何らかの動作を実行する命令は%
\term{プリミティブ}%
\index[widx]{プリミティブ@プリミティブ}%
と%
\term{マクロ}%
\index[widx]{マクロ@マクロ}%
に大別される.

プリミティブなのは{\TeX}で定義されている命令の一部であるから,
普段我々が目にしている命令のほとんどはマクロであるということになる.
これから我々が行うのも,単に「複数の命令にわかりやすい名前を付けよう」というだけにすぎない.
しかしそのためには,まず「{\TeX}が命令をどのように認識しているか」を知らなければならない.

\section{カテゴリーコードとトークン} \label{sec:categorycode}

{\TeX}では,ソース中の文字に対して%
\Cref{tab:chap::simplemacro_sec::analysis_subsec::token_category}に示す0から15までの
\index[widx]{カテゴリーコード@カテゴリーコード}%
\term{カテゴリーコード}という整数値が割り当てられている.

和文のことを無視すれば,ソース中の文字とカテゴリーコードとの対応は以下のようになっている.

\begin{table}[htbp]
	\centering
	\caption{{\TeX}におけるカテゴリーコードとその役割,および通常各カテゴリーコードに属する文字,
		ここでは和文文字は一切考慮していない.}
	\label{tab:chap::simplemacro_sec::analysis_subsec::token_category}
	\begin{tabular}{ccc}
		\toprule
		カテゴリーコード & 分類                 & 属する文字                                  \\ \midrule
		0                & 制御綴の開始         & 「\cs{}」                                   \\
		1                & グルーピング(開始) & 「\cmd{\textbraceleft}」                    \\
		2                & グルーピング(終了) & 「\cmd{\textbraceright} 」                  \\
		3                & 数式の区切り         & 「\cmd{\textbraceright}」                   \\
		4                & アライメントタブ     & 「\lstinline|&|」                  \\
		5                & 行末                 & 「\lstinline|^^M|」(文字コード13)  \\
		6                & パラメータ文字       & 「\lstinline|#|」                  \\
		7                & 上付き文字(数式中) & 「\lstinline|^|」                  \\
		8                & 下付き文字(数式中) & 「\lstinline|_|」                  \\
		9                & 無視される文字       & 「\lstinline|^^@|」(文字コード0)   \\
		10               & 空白文字             & 「\cmd{\textvisiblespace}」(文字コード32) \\
		11               & 英字                 & 英語アルファベット                          \\
		12               & 記号                 & 数字と「\cmd{!}」,「\cmd{?}」他多数        \\
		13               & アクティブ文字       & 「\lstinline|~|」                  \\
		14               & コメント開始         & 「\lstinline|%|」                  \\
		15               & 無効文字             & 「\lstinline|^^?|」(文字コード127) \\
		\bottomrule
	\end{tabular}
\end{table}

\Cref{tab:chap::simplemacro_sec::analysis_subsec::token_category}には「\lstinline|^^|」から始まる奇妙な文字もあるが,
ひとまず気にする必要はない.また,カテゴリーコード10の「\cmd{\textvisiblespace}」は単なるスペースであると考えてよい.

基本的に,カテゴリーコード11と12以外の文字はソース中で特別な役割を果たす文字であり,そのまま打ち込んでも表示されない.

{\TeX}は,ソースとして受け取ったテキストファイルを%
\index[widx]{トークン@トークン}%
\term{トークン}と呼ばれる塊に分解し,それをもとに動作する.
何を1つのトークンとみなすかは,おおむね次のような規則に従っていると考えることができる:
\begin{enumerate}
	\item カテゴリコード0の文字の直後にカテゴリーコード11の文字がある場合,
	      そこから続くすべてのカテゴリーコード11の文字をまとめて1つのトークンとみなす.
	\item このとき,そのトークン直後のカテゴリーコード10の文字は(いくつ続いても)無視される.
	\item カテゴリーコード0の文字の直後にカテゴリーコード11以外の文字がある場合,その文字までを1つのトークンとみなす.
	\item それ以外の場合,その文字1つを単独のトークンとみなす.
\end{enumerate}

\begin{Ex} \label{Ex:chap:simplemacro_sec::analysis_subsec::token_token}
	ソース中にある.「\cmd{\cs{LaTeX}3\textvisiblespace is\textvisiblespace good.}」という
	文字列を考える(「\cmd{\textvisiblespace}」は空白を表す).
	最初にカテゴリーコード0の「\cs{}」が存在していることから,そこから続く「\lstinline|\LaTeX|」までが1つのトークンとみなされる.
	そしてその直後に続くトークンは「\lstinline|3|」である.
	さらに続く英字や空白は1文字につき1つのトークンとみなされる.
	最後のトークンは「\cmd{.}」である.
	すなわち,上の文字列は「\cs{LaTeX}」「\lstinline|3|」「\cmd{\textvisiblespace}」
	「\cmd{i}」…「\cmd{d}」「\cmd{.}」という11個のトークンに分解される.
\end{Ex}

トークンのうち,カテゴリーコード0の文字(すなわちバックスラッシュ「\cs{}」)から始まるものを%
\term{制御綴}%
\index[widx]{制御綴}%
と呼ぶ.
{\LaTeX}においては,何らかの動作を行うことが期待される命令を%
\term{コマンド}%
\index[widx]{コマンド@コマンド}%
と呼ぶ.コマンドの中で主要なのは,まさにこの制御綴である.

ところで,これまで述べてきたことからもわかるように,
カテゴリーコード0の文字から続くカテゴリーコード11の文字は,まとめて1つのトークン,特に制御綴とみなされる.
このことからわかるのは,\emph{制御綴の名前に使えるのはアルファベットの大文字・小文字のみである}ということである%
\footnote{%
	和文対応環境では,和文文字も制御綴の名前に使えることがある.よく使う例として\cs{today}コマンドの書式を変更する%
	\index[cidx]{西暦@\cs{西暦}}%
	\cs{西暦}コマンドと%
	\index[cidx]{和暦@\cs{和暦}}%
	\cs{和暦}コマンドが挙げられる.%
}%
.
しかし,「\cs{\#}」や「\cs{\$}」のような制御綴の存在を考えると,このルールはおかしく見える.
このような制御綴が存在しているのは,カテゴリーコード12の文字1つだけの場合は例外的に制御綴の名前に使えるということにしているからである.
従って,「\lstinline|I watch TV\@.|」中のトークン「\lstinline|\@|」は制御綴とみなされる.
そして,その直後に空白があれば,それは無視されるわけではなく空白文字トークンとして認識されるのである.

各文字に割り当てられたカテゴリーコードは変更することができるが,
エンドユーザーとしてはあまり行うべきではないということはいうまでもないだろう.
しかし,カテゴリーコード12の文字「\lstinline|@|」だけはカテゴリーコード11に変更することがよくある.
そのためのコマンドとして,%
\index[cidx]{makeatletter@\cs{makeatletter}}%
\cs{makeatletter}コマンドと%
\index[cidx]{makeatother@\cs{makeatother}}%
\cs{makeatother}コマンドが用意されている.
\cs{makeatletter}コマンドを記述した場所以降では「\cmd{@}」がカテゴリーコード11に変更され制御綴の名前に使えるようになり,
\cs{makeatother}コマンドを記述した場所以降では「\cmd{@}」がカテゴリーコード12に戻る.
この仕組みは,ユーザー側に触れてほしくない内部でのみ使われるコマンドを定義するのに役に立つ.

\section{引数の処理} \label{subsec:chap::simplemacro_sec::analysis_argument}

{\TeX}で扱う制御綴には,引数をともなうものとそうでないものがあった.
そして,どこからどこまでが引数であるかはグルーピングを表す文字「\cmd{\textbraceleft}」と「\cmd{\textbraceright}」
で指定するのであった.

引数が複数のトークンからなる場合,グルーピングによってその範囲を指定しなければならない.
一方で,引数がただ1つのトークンからなる場合,グルーピングによる範囲指定は省略できる
(グループ指定を省略した場合,直後のトークンが引数とみなされる).

\begin{Ex} \label{Ex:chap::simplemacro_sec::analysis_subsec::argument_argument}
	それぞれの書き方で何が\cs{textbf}コマンドの引数とみなされているかに注意しよう.
	\cs{textbf}コマンドは引数を1つとる制御綴であった.
    \inputonly{samplecode/argument}  
    出力は,上から順に「\textbf\LaTeX」「\textbf LaTeX」「\textbf{LaTeX}」となる.
\end{Ex}

マクロを作成していて奇妙な挙動が見られた場合,引数指定が適切であるかどうかも考えておく必要があるだろう.


\section{展開と実行} \label{subsec:chap::simplemacro_sec::analysis_expand}

トークンには,処理された結果別のトークン列に変換されるものがある.
この変換処理を%
\index[widx]{てんかい@展開}%
\term{展開}という.展開できるトークンは%
\index[widx]{てんかい@展開!てんかいかのう@---可能}%
\term{展開可能}であるという.
あるトークンの展開結果として現れたトークンは,展開可能であれば再び展開される.
そして,展開可能でないトークンになるまで展開操作は続く.
展開可能でないトークンが現れた場合,そのトークンによって定められた何らかの処理が行われることになる.
これを%
\index[widx]{じっこう@実行}%
\term{実行}という.

当然のことながら,制御綴のうちで展開可能でないものはプリミティブに限る
(が,プリミティブでも展開可能であるものは存在する).
我々が文書作成でよく使う制御綴はほとんどすべて展開可能である.

以上の事項を念頭に置きながら,マクロの作成方法について学んでいく.
\chapter{マクロの作り方} \label{chap:macromethod}

\section{新しいコマンドの定義} \label{sec:defcmd}

新しいコマンドを定義する場合,%
\cs{NewDocumentCommand}%
\index[cidx]{NewDocumentCommand@\cs{NewDocumentCommand}}
コマンドを用いるのがよい%
\footnote{%
もしかしたら%
\cs{def}%
\index[cidx]{def@\cs{def}}%
や%
\cs{newcommand}%
\index[cidx]{newcommand@\cs{newcommand}}
%
の方が見慣れているかもしれないが,基本的には
\cs{NewDocumentCommand}コマンドを用いるべきである.
\cs{def}は{\TeX}レベルの制御綴で,\cs{newcommand}は
\cs{NewDocumentCommand}よりも機能が劣っているとみなせるからである.
}%
.

\cs{NewDocumentCommand}コマンドは新しめの環境であれば追加パッケージを読み込むことなく使えるが,
そうでない場合は
\package{xparse}%
\index[pidx]{xparse@\package{xparse}}%
パッケージを読み込めば使用可能になる.

\cs{NewDocumentCommand}コマンドを使って新しいコマンドを定義する場合,(普通は)プリアンブルで以下のように記述する:
\inputonly{samplecode/newdocumentcommand}
あるいは,最初の「\cs{}\meta{command name}」の部分をかっこでくくって
\inputonly{samplecode/newdocumentcommandwithbrace}
のように記述してもよい.
どちらも意味は同じで
「\meta{argument specification}という引数仕様のコマンド\cs{}\meta{command name}を,\meta{definition}と定義した」ということになる.
これにより,ソース中の「\cs{}\meta{command name}」は「\meta{definition}」に置き換えて解釈される.

\cs{NewDocumentCommand}コマンドが引数として要求している「引数仕様」とは,
定義したい「\cs{}\meta{command name}」がどのような引数をもつかを記述するものである.
よく使うものを\cref{tab:specification}に示す.

\begin{table}[htbp]
	\centering
	\caption{\cs{NewDocumentCommand}コマンドでよく使う引数仕様の一覧,複数の引数をもつコマンドを定義する場合,順番に並べて記述する.}
	\label{tab:specification}
	\begin{tabular}{ccc}
		\toprule
		記号 & 意味 \\ \midrule
        \lstinline|m| & 必須引数 \\
        \lstinline|o| & オプション引数(既定値なし) \\
        \lstinline|O|\metamarg{default value} & オプション引数(既定値が\meta{default value}) \\
        \lstinline|s| & スター(「\lstinline|*|」)の有無を判定 \\
		\bottomrule
	\end{tabular}
\end{table}

指定した引数には\meta{definition}の部分で,登場順に「\lstinline|#1|」や「\lstinline|#2|」のようにアクセスできる.

引数仕様に「\lstinline|s|」を指定した場合,
この引数には「\lstinline|*|」の有無を示すフラグが付与される.
使用時には「\cs{}\meta{cmd}\cmd{*}\metamarg{second argument}」
のように記述する.第1引数が「\lstinline|*|」のみ指定可能なオプション引数になったイメージである.

引数仕様に「\lstinline|s|」を指定した場合,その引数を使って処理を分岐させることができる.
そのために役に立つのが
\cs{IfBooleanTF}%
\index[cidx]{IfBooleanTF@\cs{IfBooleanTF}}%
コマンドである.
このコマンドは
\inputonly{samplecode/IfBooleanTF}
のように使用する.「\lstinline|*|」がある場合には\meta{true expression}が,
ない場合には\meta{false expression}がそれぞれ実行される.
このようにして,あるコマンドの「亜種」とも呼ぶべき\lstinline|*|付きのコマンドを簡単に定義できる.

{\LaTeX}における\cs{providecommand}コマンドのように,
「そのコマンドが定義済みでない場合のみ定義を行い,定義済みの場合は何もしない」という処理を行うための
\cs{ProvideDocumentCommand}%
\index[cidx]{ProvideDocumentCommand@\cs{ProvideDocumentCommand}}%
コマンドも存在する.
使い方はまったく同じで
\inputonly{samplecode/providedocumentcommand}
のように使用する.「\meta{argument specification}」の部分も\cs{NewDocumentCommand}コマンドとまったく同じである.

\section{新しい環境の定義} \label{sec:defenv}

新しい環境を定義する場合,
\cs{NewDocumentEnvironment}%
\index[cidx]{NewDocumentEnvironment@\environment{NewDocumentEnvironment}}%
コマンドを用いる.
このコマンドは
\inputonly{samplecode/newdocumentenvironment}
のように使用する.これにより「\meta{argument specification}という引数仕様の\meta{environment name}環境を,開始時に
\meta{begin process}を,終了時に\meta{end process}を行う環境として定義した」ということになる.
引数仕様の部分はもちろん\cs{NewDocumentCommand}のものとほぼ同じである%
\footnote{%
引数仕様の「\lstinline|s|」は,\cs{NewDocumentEnvironment}コマンドで使用した場合には挙動が異なる.
詳細は\package{xparse}パッケージのドキュメントを参照せよ.
}%
.


\section{既存の定義の書き換え} \label{sec:renew}

既存のコマンドの定義を書き換える場合,
\cs{RenewDocumentCommand}%
\index[cidx]{RenewDocumentCommand@\cs{RenewDocumentCommand}}%
コマンドを用いるとよい.
使い方はやはり\cs{NewDocumentCommand}コマンドとまったく同じである.
異なるのは,\cs{NewDocumentCommand}コマンドでは定義対象となるコマンドが未定義であることが要請されていたが,
\cs{RenewDocumentCommand}コマンドではその逆で,定義対象となるコマンドが定義済みであることを要請する.

\cs{NewDocumentEnvironment}コマンドについても同様に,既存の環境の定義を書き換えるための%
\cs{RenewDocumentEnvironment}%
\index[cidx]{RenewDocumentEnvironment@\cs{RenewDocumentEnvironment}}%
コマンドが存在する.

「定義対象が定義済みであろうとなかろうと問答無用で定義を行う」という処理を行うための%
\cs{DeclareDocumentCommand}%
\index[cidx]{DeclareDocumentCommand@\cs{DeclareDocumentCommand}}%
コマンドや%
\cs{DeclareDocumentEnvironment}%
\index[cidx]{DeclareDocumentEnvironment@\cs{DeclareDocumentEnvironment}}%
コマンドも存在する.
しかし,これらは「定義済みとは知らずにうっかり上書きしてしまった」というリスクがあるため,
どうしても必要な場合以外は使用を控えるのが望ましい.
\chapter{実践例} \label{chap:ex}

\section{単純な例} \label{sec:simple}

まずは単純な例から始めよう.

\begin{Ex} \label{Ex:cppnewdocumentcommand}
    \Cref{Ex:cpp}で述べた\cs{cpp}コマンドは次のように定義できる:
    \inputonly{samplecode/cppdef}
    \cs{NewDocumentCommand}コマンドの第2引数である引数仕様を空にしておくことで,
    引数を持たないコマンドを定義できる.
    本文中では「\cs{cpp}」とのみ記述すればよいので,保守性や可読性の向上が見込める.
\end{Ex}

\Cref{Ex:changedef}で述べた\cs{emph}コマンドの再定義を考えよう.

\begin{Ex} \label{Ex:emphbfdef}
    \cs{emph}コマンドの振る舞いを\cs{textbf}コマンドと同じようにしたければ,
    いっそのこと\cs{textbf}コマンドと同じ挙動にしてしまえばいい.
    つまり,
    \inputonly{samplecode/emphbfdef}
    のようにしてしまえばいい.
    こうすれば,\cs{emph}\marg{hoge}と書くのと\cs{textbf}\marg{hoge}と書くのが等価になる.
    振る舞いを\cs{textbf}コマンドと同じにしたうえで本文中では「\cs{emph}\marg{hoge}」のままにしておくことで,
    「この箇所は強調である」という意図がはっきりと残る.
\end{Ex}

\section{数式が関わる例} \label{sec:math}

数式が関わる文脈では,構造化マークアップはよりその力を発揮する.

\begin{Ex} \label{Ex:number}
    実数全体の集合\(\symbb{R}\)を記述する際,よく見るのは(\package{amsmath}パッケージ読み込み時では)「\cs{mathbb}\marg{R}と記述せよ」
    というものである.しかし「\cs{mathbb}\marg{R}」では「Rを黒板太字で出力せよ」という意図しか伝わらない.
    そこで
    \inputonly{samplecode/realnumber}
    のように定義し,本文中では「\cs{realnumbers}」と記述することで「実数全体の集合を出力せよ」という意図がはっきりと伝わる.
    さらに,例えば
    \package{unicode-math}%
    \index[pidx]{unicode-math@\package{unicode-math}}%
    パッケージを採用することを後から決めた場合,\cs{realnumber}コマンドの定義を
    \inputonly{samplecode/realnumbersymbb}
    と変更するだけで修正が完了する.本文中で何度実数全体の集合が登場しようと
    修正は1箇所のみである.
\end{Ex}

\begin{Ex} \label{Ex:paren}
    数学において,かっこは文脈に応じてさまざな意味で使われる.
    例えば,\(x\)に関数\(f\)を適用した結果を表す「\(f(x)\)」や,項の構成順序を表す「\(a(x + y)\)」である.
    後者に関しては明確な意味があるとは言い難いが,前者には「関数適用」という確固たる意味を有する.
    そこで,以下のように定義してみよう:
    \inputonly{samplecode/funcparen}
    \lstinline|*|の有無でかっこの大きさを\cs{left}と\cs{right}で調整するかどうかを指定できる.

    このように定義することで,関数適用\(f(x)\)が「\lstinline|\apply{f}{x}|」と記述できるようになる.
    同時に,コマンド「\cs{paren}」も合わせて定義することで,項の構成順序を表す「\(a(x + y)\)」は
    「\lstinline|a\paren{x + y}|」のように記述できる.
    \cs{paren}コマンドを定義するのは,意図をはっきりさせるというよりむしろ複雑な実装を隠蔽するという意味が強い.
\end{Ex}

\Cref{Ex:paren}で出てきた%
\cs{mathopen}%
\index[cidx]{mathopen@\cs{mathopen}}%
コマンドと
\cs{mathclose}%
\index[cidx]{mathclose@\cs{mathclose}}%
コマンドについて補足しておこう.
これらのコマンドはともに引数を1つとり,その引数をそれぞれ数式における
開きかっこと閉じかっことして認識させるコマンドである.
これらで\cs{left}と\cs{right}をはさむことにより,
\cs{left}と\cs{right}に起因する余白を抑制することができる.

ところで,\cs{mathopen}や\cs{mathclose}は,
構造化マークアップがマクロを組むことだけではないことを示す好例である.

\begin{Ex} \label{Ex:mathopenclose}
    左半開区間は,
    \(\mathopen{(}a, b \mathclose{]}\)と書かれることもあれば\(\mathopen{]}a, b\mathclose{]}\)
    と書かれることもある.
    とくに後者をきちんと「\lstinline|\mathopen{]} a, b \mathclose{]}|」と書くことで,
    本来閉じかっことして振る舞う「\(\mathclose{]}\)」を開きかっことして認識させることができる.
\end{Ex}

構造化マークアップとは,自分の意図を明瞭になるようにソースを記述することを志向したマークアップの技法である.
したがって,それを実現する手段はなんでもよい.
マクロを組むのは多くの場合に有用な解決策であるが,それが唯一ではないのである.
%\include{contents/equation}
%\include{contents/tipsexample}
%

\backmatter
\nocite{*}
\printbibliography[title=参考文献,heading=bibintoc]


\printindex[cidx]
%\printindex[pidx]
%\printindex[eidx]
\printindex[widx]

\include{contents/colophon}

\end{document}
\chapter{はじめに}

現在,{\TeX}/{\LaTeX}は自然科学,特に数学や物理学などの数式が多く含まれるような
分野において,レポートや論文のような文書を作成する際には事実上必須のツールとなっている.
しかも,レポートや論文だけでなくセミナーの資料やちょっとしたネット上の解説記事も
{\TeX}/{\LaTeX}で作成されることが多くなっている.
このように,{\TeX}/{\LaTeX}は特に自然科学を学ぶ者にとっては馴染みのあるツールであり,
当該分野の知識に加えてその使い方を習得する必要に駆られている.

では彼らがどのように{\TeX}/{\LaTeX}の使い方を習得するかといえば,
広く出版されている入門書を手に取る場合もあるけれども,
おおよそ研究室やサークルの先輩からの口伝や,
インターネット上の解説記事を頼りにした独学が多いのではないだろうか.
もちろん彼らは{\TeX}/{\LaTeX}のエキスパートになりたいわけではなく,
自分の要求に堪えるレベルで文書が作成できればそれで十分なのではあるが,
そのためかあまり推奨されないような使い方をしている場面にしばしば遭遇する.
具体例を挙げると,
\begin{itemize}
	\item \cs{quad}や\cs{textbf}などの「見た目系コマンド」の本文中での多用,
	\item \ltxclass{jarticle}や\ltxclass{jreport}などの時代遅れのドキュメントクラスの利用,
	\item \env{eqnarray}環境や\lstinline{$$...$$}のような時代遅れの数式環境の利用
\end{itemize}
などである.

これらの問題のうち,後者2つに関しては解説を見かける機会は非常に多い.
時代遅れのドキュメントクラスや環境の使用は,出力がおかしくなったりそもそもタイプセットができないといった
致命的な問題を引き起こすからであろう.
しかし,1つ目の問題に対しては後者2つに比べれば解説を見かけることは多くない.
「見た目系コマンド」が本文中で多用されたからといって,出力がおかしくなるような致命的な問題は生じないからであろう.

{\TeX}/{\LaTeX}はマークアップシステムであるから,文書の意味内容を表す「構造」とそれらの構造がどのように紙面に表現されるかの「体裁」は
できる限り分離されるべきである.
この「構造と体裁をできる限り分離してソースを書こう」という姿勢でなされるマークアップは構造化マークアップと呼ばれ,
特にWebページを作成する際には必須とみなされている%
\footnote{%
Webページを作成する際に構造化マークアップが必須となるのはブラウザの多様化や検索エンジンとの関係が大きく,
筆者が{\LaTeX}で構造化マークアップを勧める理由とはかなり異なるが.
}%
.

本書は{\LaTeX}界隈にもこの流れを取り込むべく執筆された本である.
具体的には,{\TeX}/{\LaTeX}の基本的な事項を習得した初心者が
構造化マークアップを意識した美しいコードを書けるようになることが目標である.
従って,{\TeX}/{\LaTeX}の基本的な事項は習得済みであることを想定している.
{\TeX}/{\LaTeX}でレポートや論文を作成した経験があればあまり問題ないと思われる.
まずは新しいコマンドや環境の作成方法を述べ,
それを使って文章や数式の意味内容を効果的に表現する例を挙げる.

本書で挙げたテクニックを習得できれば,ソースコード中のコマンドの意図が分かりやすくなり可読性が向上するだけでなく,
汎用的に使えるコマンドを切り出して別ファイルに保存し,それを別の文書に使いまわすことも容易となる%
.本書の読者がこのような恩恵を実感できるようになれば何よりである.

本書では,環境に対する依存性が低い話題が中心であるが,具体的な実装面では2022年12月ごろにインストールされた{\TeXLive} 2022を想定する.
それでも,違いが生じうるのはおおよそ「\package{xparse}パッケージを読み込む必要があるかどうか」程度であろう.

なお,本書ではソース中の空白文字を可視化するため,必要な場合には「\cs{\textvisiblespace}」という記法で空白文字を表現することがある.
また,ソース中で「\cs{textbf}\metamarg{argument}」のように山括弧で囲まれたイタリック体の文字が登場した場合,
それは本当に「\lstinline|<argument>|」と記述するわけではなく,状況に応じて適切なものに置き換えることを意図して書かれている.
このことに注意しながら読み進めてほしい.